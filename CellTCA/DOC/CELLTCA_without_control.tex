\documentclass[a4paper,10pt]{article}
\usepackage{color,pdfcolmk}
\usepackage{afterpage}
\setlength{\hoffset}{-18pt}        
\setlength{\oddsidemargin}{0pt} % Marge gauche sur pages impaires
\setlength{\evensidemargin}{9pt} % Marge gauche sur pages paires
\setlength{\marginparwidth}{54pt} % Largeur de note dans la marge
\setlength{\textwidth}{481pt} % Largeur de la zone de texte (17cm)
\setlength{\voffset}{-18pt} % Bon pour DOS
\setlength{\marginparsep}{7pt} % Séparation de la marge
\setlength{\topmargin}{0pt} % Pas de marge en haut
\setlength{\headheight}{13pt} % Haut de page
\setlength{\headsep}{10pt} % Entre le haut de page et le texte
\setlength{\footskip}{27pt} % Bas de page + séparation
\setlength{\textheight}{708pt} % Hauteur de la zone de texte (25cm)
\renewcommand{\arraystretch}{3}

\author{KOPP Arnaud}

\title {TCA Plate Analysis using CELLTCA without controls and R}

\begin{document}
\maketitle


\section*{prerequisite}
We need to know one RefSamp (neg samp), one Pos Samp (posSamp) and the columns ( of the data ) to analyze.

\bigskip

R PACKAGE TO INSTAL:
\begin{itemize}
  \item limma (BioConductor)
  \item prada (BioConductor)
  \item bioDist (BioConductor)
  \item grid
  \item e1071
  \item ggplot2 (graphics)
  \item reshape2 (graphics)
  \item fields
\end{itemize}

\section*{Operation performed}

\subsection*{Quality control}
Check if gene is found in all plate, if not, it will be removed.
Check if negative and positive control have non identical distribution.

We can calculate the similarity/pseudo-distance between two distributions (replicates). If a replicate is considered as a 
outliers, it will be excluded. 

\subsection*{Normalization}
If only one replicate, no intra plate normalization is available. Check if in case a gene is found in one plate and not
 in another one, we remove it to prevent from performed normalization with false data.
A quantile normalization is performed, ( "scale", "quantile", "Aquantile", "Gquantile", "Rquantile" or "Tquantile" 
or "cyclicloess") are available. function = normalizeBetweenArrays from limma package

Pooling: the data is organized on a column basedeach line will consist of gene id and column values As each each column
represents the same distribution, we only need one of them. Preferably, one without na or with less na entries.



\subsection*{P-Values}
Mann-Whitney test is performed.
Mann-Whitney test to check that positive control distribution is significantly greater than negative control distribution
The null hypothesis is that neg and pos are identical, if pvalue < 0.05 then null hypo are reject (neg and pos are nonindentical)

\subsection*{Cutoff}
Cutoff is systematicely calculed, it need neg and pos reference.
First, it calculate density for each gene for neg and pos reference. The algorithm used in density disperses the mass of the
empirical distribution function over a regular grid of at least 512 points (2048 for us) and then uses the fast Fourier 
transform to convolve this approximation with a discretized version of the kernel and then uses linear approximation to evaluate
the density at the specified points. 


\subsection*{Calcul performed}

\begin{tabular}{|l|l|}
\hline

percentPValues &   \\
Viability & =$\frac{(c-\mu_p-3\sigma_p)}{|\mu_n-\mu_p|}$ \\
lzfactor (plaque) & =$1-\frac{3\sigma_p+3\sigma_n}{|\mu_p-\mu_n|}$ \\
zfactor (screen) &  =$1-\frac{3\sigma_p+3\sigma_n}{|\mu_p-\mu_n|}$\\
Percent of Control & $\frac{\% Cellules positive}{\% Cellules positives dans le controle pos}$  \\
Percent of Control SD &   \\
p-value ( $\neq \alpha$) &     \\

\hline
\end{tabular}

\bigskip



\subsection*{Hit identification}

\section*{Organization of the results}
The results are organized according to the various of TCA plates submitted. Each plate may have
several replicates. In case of more than one replicate, a quantile-quantile inter-plate normalization
is performed.

 
Each user-defined analysis column of measures from the input-data gives rise to six plots : a plot showing
the positive control and the negative control separated by an estimated cutoff for positive cells identification, a barplot 
that shows the percentage of cell intensity values greater than the estimated cutoff, a barplot 
to visualize the control percents (percentage of positive cells in any well divided by the percentage
of positive cells in the negative control), a barplot representing the average cell intensity values per well, a plot that shows the 
spatial distribution of the average cell intensity values per well, and a plot to visualize the spatial distribution 
of the Coefficient of Variation values per well. These plots are generated for each plate of the screening. Additionally,
some plots will be generated to give some insights of the quality of the screening as a whole. The first one represents the average medians
caculated for every well over the screening. This can help identifying
systematic effects occuring at the same position of the plates. The next one is a cell count distribution
over the screening. This plot reflects the variation of cell number per well over the screening. By comparing 
the cell counts distribution plot of the positive toxicity control with respectively the distributions of cell counts of 
the toxicity negative control and the samples, anomalies related to cell number variation can be detected.

 \subsection*{Genes Selection}
 The genes are selected using testing hypotheses procedures. The wells are charaterized by their percent
 of control. Given $x_{p}$ the percent of positive cells in a well and $c_{p}$ the percent of positive 
 cells in the negative control, genes are selected when $x_{p}/c_{p}$ is statistically significantly greater
 than $\alpha$ with $\alpha\geq1$ or less than $\alpha$ with $\alpha\leq1$.
 $\alpha$ is a sensitivity parameter. By increasing $\alpha$, we will have more confident
 genes that have a greater percent of positive cells than the negative control. In the output, we
 report p-values for three different values of $\alpha$ depending uppon samples size difference with the negative control.
 Aside from p-values there will be additional information to select which genes to prioritize. A table 
 is provided. The table will also contain a viability value calulated using
 the available positive toxicity control and a negative toxicity control cell counts distributions. The viability value
 is calculated using a Z-factor like measure: 
 \[\varphi(c)=\frac{(c-\mu_p-3\sigma_p)}{|\mu_n-\mu_p|}\]
 $c$ is the number of cells in the well of interest, $\mu_p$ is the average cells number in the positive toxicity
 control, $\mu_n$ is the average cells number in the negative control, $\sigma_p$ is the standard deviation of the
 cells number distribution in the postive control and $\sigma_n$ is the standard deviation of the cells number 
 distribution in the negative control. These parameters are calculated using normal approximation of the cell counts 
 distribution. The Z-factor as quality metric can be used as a threshold for non-toxic gene selection.
 \[Z-factor=1-\frac{3\sigma_p+3\sigma_n}{|\mu_p-\mu_n|}\]     
 Basically, the Z-factor is used to quantify the quality of an experiment. High value of Z-factor corresponds to
 an experiment where the positive and the negative controls are well separated, otherwise the Z-factor value is low. 
 It results that the genes with a viability value greater than the Z-factor may be considered as non toxic, meaning that 
 their number of cells is similar to the negative toxicity control condition. Based on proximity of cell counts,
 to negative and positive toxicity controls, genes with negative values of viability are considered as toxic and therefore
 are labelled as toxic.


A Mann-Witney test is performed to check how different positive and negative 
controls are. We expect to see a shift to the right when compare positive and negative 
controls distribution. The p-Value should be at most 0.01 depending on the sample size.



\end{document}
