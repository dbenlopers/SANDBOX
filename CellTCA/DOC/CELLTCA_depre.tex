\documentclass[a4paper,10pt]{article}
\usepackage{color,pdfcolmk}
\usepackage{amssymb,amsmath}
\usepackage{afterpage}
\setlength{\hoffset}{-18pt}        
\setlength{\oddsidemargin}{0pt} % Marge gauche sur pages impaires
\setlength{\evensidemargin}{9pt} % Marge gauche sur pages paires
\setlength{\marginparwidth}{54pt} % Largeur de note dans la marge
\setlength{\textwidth}{481pt} % Largeur de la zone de texte (17cm)
\setlength{\voffset}{-18pt} % Bon pour DOS
\setlength{\marginparsep}{7pt} % Séparation de la marge
\setlength{\topmargin}{0pt} % Pas de marge en haut
\setlength{\headheight}{13pt} % Haut de page
\setlength{\headsep}{10pt} % Entre le haut de page et le texte
\setlength{\footskip}{27pt} % Bas de page + séparation
\setlength{\textheight}{708pt} % Hauteur de la zone de texte (25cm)


\author{KOPP Arnaud}

\title {TCA Plate Analysis using CELLTCA and R}

\begin{document}
\maketitle




\section{prerequisite}
We need to know one RefSamp (neg samp), one Pos Samp (posSamp) and the columns ( of the data ) to analyze.

\bigskip

R PACKAGE TO INSTAL:
\begin{itemize}
  \item limma (BioConductor)
  \item prada (BioConductor)
  \item bioDist (BioConductor)
  \item grid
  \item e1071
  \item DAAG
  \item fitdistrplus
  \item FactoMineR
  \item ggplot2 (graphics)
  \item reshape2 (graphics)
  \item fields
\end{itemize}


\subsection*{Hit identification}


\section{Methods}
After plate-wise quantile-quantile normalization, gene are selected according to how they differ from
the negative condition. We name positive cells, cells that are more likely to be found in the positive 
control condition and negative cells, the ones that are similar the cells in the negative control
condition. Genes are selected based on the their difference to the negative control in terms of
number of positive cells. To quantify the likelyhood of a cell to be positive or negative, we 
used a supervised learning approach. More specifically, we apply a logistic regression using 
the data from the positive and the negative controls. For every cell $c_{i}$, we define :
\begin{center}
  $\mathbb{P}(Positive | c_{i}) = \frac{1}{1+e^{-(\theta_{0}+c_{i}\theta_{1})}}$
\end{center}  
the pobability of a cell to be positive given its intensity value. $\theta_{0}$ and $\theta_{1}$ are 
learned from the data. The dynamic range between the positive and
the negaticve is addressed by predictive accuracy of the model using a 10-fold cross-validation. 
Depending on the quality of tha ssay data, we expect that the predictive accuracy to be at least 70\%.
In other words, 30\% of training data are misclassified.
 
\paragraph{}
For a specific gene, every cell has now a probabiluity value to belong to the positive control. These
values can be modeled by a beta distribution whose parameters $\alpha$ and $\beta$ can be estimated by the cell data taken
from that gene condition. To be able to compare the genes to the negative control, we assign to every
gene $g_{i}$ the average of its cell values:   
\begin{center}
$\bar{g}_{i} = \frac{1}{n}\sum_{j=1}^{n}p_{ij}$
\end{center}
where the $p_{ij}$ with $j=1 .. n$ are the cell probabilities for that gene. Gene selection is done using
the relative risk associated with every gene $g_{i}$. It defines as :
\begin{center}
$r_{i} = \frac{\bar{g}_{i}}{\bar{g}_{c}} $
\end{center}
More specifically, we use the log value of $r_{i}$. As $r_{i}$, $\bar{g}_{i}$ and $\bar{g}_{c}$ are estimators, 
using the delta method, we have:
\begin{center}
$\mathbb{E}(log(\bar{g}_{i})) = log(\frac{\alpha_{i}}{\alpha_{i}+\beta_{i}})$ and $\mathbb{V}ar(log(\bar{g}_{i}))=\frac{\alpha_{i}\beta_{i}}
{n_{i}(\alpha_{i}+\beta_{i})^{2}(\alpha_{i}+beta_{i}+1)} $
\end{center}
and 
\begin{center}
$\mathbb{E}(log(\bar{g}_{c})) = log(\frac{\alpha_{c}}{\alpha_{c}+\beta_{c}})$ and $\mathbb{V}ar(log(\bar{g}_{c}))=\frac{\alpha_{c}\beta_{c}}
{n_{c}(\alpha_{c}+\beta_{c})^{2}(\alpha_{c}+beta_{c}+1)} $
\end{center}
Under the null hypothesis that there is no difference between the gene $i$ and the negative control, we have :
\begin{center}
$\mathbb{E}(log(\bar{g}_{i})-log(\bar{g}_{c})) = 0$
\end{center} 
Therefore, given the number of cells per gene is sufficiently large, p-values can be calculated using the value :
\begin{center}
 $Zscore = \frac{log(\bar{g}_{i})-log(\bar{g}_{c})}{\sqrt{\mathbb{V}ar(log(\bar{g}_{i}))+\mathbb{V}ar(log(\bar{g}_{c}))}}$
\end{center} 
Two p-values are reported depending the user interest:
\begin{center}
$p-value_{+} = 1 - F_{Z}(Zscore)$ and $p-value_{-} = F_{Z}(Zscore)$
\end{center}
The p-values are ajusted to FDR values.

\paragraph{}
 A viability value is calulated using
 the available positive toxicity control and a negative toxicity control cell counts distributions. The viability value
 is calculated using a Z-factor like measure:
 \[\varphi(c)=\frac{(c-\mu_p-3\sigma_p)}{|\mu_n-\mu_p|}\]
 $c$ is the number of cells in the well of interest, $\mu_p$ is the average cells number in the positive toxicity
 control, $\mu_n$ is the average cells number in the negative control, $\sigma_p$ is the standard deviation of the
 cells number distribution in the postive control and $\sigma_n$ is the standard deviation of the cells number
 distribution in the negative control. These parameters are calculated using normal approximation of the cell counts
 distribution. The Z-factor as quality metric can be used as a threshold for non-toxic gene selection.
 \[Z-factor=1-\frac{3\sigma_p+3\sigma_n}{|\mu_p-\mu_n|}\]
 Basically, the Z-factor is used to quantify the quality of an experiment. High value of Z-factor corresponds to
 an experiment where the positive and the negative controls are well separated, otherwise the Z-factor value is low.
 It results that the genes with a viability value greater than the Z-factor may be considered as non toxic, meaning that
 their number of cells is similar to the negative toxicity control condition. Based on proximity of cell counts,
 to negative and positive toxicity controls, genes with negative values of viability are considered as toxic and therefore
 are labelled as toxic.




\end{document}
